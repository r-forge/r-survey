\documentclass{article}
\usepackage{url}
\addtolength{\topmargin}{-0.5in}
\addtolength{\textheight}{0.75in}
\addtolength{\oddsidemargin}{-0.5in}
\addtolength{\textwidth}{1in}
%\VignetteIndexEntry{Some details on two-phase variances}
\usepackage{Sweave}
\author{Thomas Lumley}
\title{Some details on two-phase variances}

\begin{document}
\maketitle
This document explains the computation of variances for totals in
two-phase designs. Variances for other statistics are computed by the
delta-method from the variance of the total of the estimating
functions.

The variance formulas come from conditioning on the sample selected in
the first phase 
$$\textrm{var}[\hat T]=E\left[\textrm{var}\left[\hat T|\textrm{phase 1}\right]\right]+\textrm{var}\left[E\left[\hat T|\textrm{phase 1}\right]\right]$$

The first term is estimated by the variance of $\hat T$ considering
the phase one sample as the fixed population, and so uses the same
computations as any single-phase design. The second term is the
variance of $\hat T$ if complete data were available for the phase-one
sample. This takes a little more work.


The variance computations for a stratified, clustered, multistage
design involve recursively computing a within-stratum variance for the
total over sampling units at the next stage.  That is, we want to
compute 
$$s^2=\frac{1}{n-1}\sum_{i=1}^n (X_i-\bar X)$$
where $X_i$ are $\pi$-expanded observations, perhaps summed over sampling units.
A natural estimator of $s^2$ when only some observations are present in the phase-two sample is
$$\hat s^2=\frac{1}{n-1}\sum_{i=1}^n \frac{R_i}{\pi_i} (X_i-\hat{\bar X})$$
where $\pi_i$ is the probability that $X_i$ is available and $R_i$ is the indicator that $X_i$ is available.  We also need an estimator for $\bar X$, and a natural one is
$$\hat{\bar X}=\frac{1}{n}\sum_{i=1}^n \frac{R_i}{\pi_i}X_i$$

This is not an unbiased estimator of $s^2$ unless $\hat{\bar X}=\bar X$,
but the bias is of order $O(n_2^{-1})$ where $n_2=\sum_i R_i$ is the
number of phase-two observations.

If the phase-one design involves only a single stage of sampling then
$X_i$ is $Y_i/p_i$, where $Y_i$ is the observed value and $p_i$ is the
phase-one sampling probability.  For multistage phase-one designs (not
yet implemented) $X_i$ will be more complicated, but still feasible to
automate.

This example shows the unbiased phase-one estimate (from Takahiro
Tsuchiya) and the estimate I use, in a situation where the phase two
sample is quite small.

First we read the data
\begin{verbatim}
rei<-read.table(textConnection(
"  id   N n.a h n.ah n.h   sub  y
1   1 300  20 1   12   5  TRUE  1
2   2 300  20 1   12   5  TRUE  2
3   3 300  20 1   12   5  TRUE  3
4   4 300  20 1   12   5  TRUE  4
5   5 300  20 1   12   5  TRUE  5
6   6 300  20 1   12   5 FALSE NA
7   7 300  20 1   12   5 FALSE NA
8   8 300  20 1   12   5 FALSE NA
9   9 300  20 1   12   5 FALSE NA
10 10 300  20 1   12   5 FALSE NA
11 11 300  20 1   12   5 FALSE NA
12 12 300  20 1   12   5 FALSE NA
13 13 300  20 2    8   3  TRUE  6
14 14 300  20 2    8   3  TRUE  7
15 15 300  20 2    8   3  TRUE  8
16 16 300  20 2    8   3 FALSE NA
17 17 300  20 2    8   3 FALSE NA
18 18 300  20 2    8   3 FALSE NA
19 19 300  20 2    8   3 FALSE NA
20 20 300  20 2    8   3 FALSE NA
"), header=TRUE)
\end{verbatim}

Now, construct a two-phase design object and compute the total of \verb=y=
\begin{Schunk}
\begin{Sinput}
> des.rei <- twophase(id = list(~id, ~id), strata = list(NULL, 
+     ~h), fpc = list(~N, NULL), subset = ~sub, data = rei)
> tot <- svytotal(~y, des.rei)
\end{Sinput}
\end{Schunk}

The unbiased estimator is given by equation 9.4.14 of S\"arndal, Swensson, \& Wretman.
\begin{Schunk}
\begin{Sinput}
> rei$w.ah <- rei$n.ah/rei$n.a
> a.rei <- aggregate(rei, by = list(rei$h), mean, na.rm = TRUE)
> a.rei$S.ysh <- tapply(rei$y, rei$h, var, na.rm = TRUE)
> a.rei$y.u <- sum(a.rei$w.ah * a.rei$y)
> a.rei$f <- with(a.rei, n.a/N)
> a.rei$delta.h <- with(a.rei, (1/n.h) * (n.a - n.ah)/(n.a - 1))
> Vphase1 <- with(a.rei, sum(N * N * ((1 - f)/n.a) * (w.ah * (1 - 
+     delta.h) * S.ysh + ((n.a)/(n.a - 1)) * w.ah * (y - y.u)^2)))
\end{Sinput}
\end{Schunk}

The phase-two contributions (not shown) are identical. The phase-one contributions are quite close
\begin{Schunk}
\begin{Sinput}
> Vphase1
\end{Sinput}
\begin{Soutput}
[1] 24072.63
\end{Soutput}
\begin{Sinput}
> attr(vcov(tot), "phases")$phase1
\end{Sinput}
\begin{Soutput}
         [,1]
[1,] 23461.05
\end{Soutput}
\end{Schunk}


\end{document}
